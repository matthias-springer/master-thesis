\chapter{Summary}
\label{sec:summary}
We presented \msname, a module system for Squeak/Smalltalk. \msname is inspired by Newspeak and based on class nesting. Top-level classes are part Smalltalk globals and nested classes can be accessed by sending the class' name as a message to the enclosing class. Nested classes effectively establish a hierarchical namespace, similar to Java pacakages, Ruby modules, or Newspeak nested classes. In contrast to Newspeak, \msname's namespace is global. In Newspeak, however, every module has its own separate namespace and can access dependencies only through a \texttt{platform} object.

\section{Modularity in Newspeak}
Arguably, Newspeak is even more modular than \msname. For example, it does not have a global namespace, which is a form of global state. In addition, Newspeak has a concept of method visibility, making it possible to enforce an interface by declaring only API methods as public methods. 

However, these benefits come at a price. Even though Newspeak evolved out of Squeak, it needs a modified virtual machine\footnote{The COG VM is the \emph{development VM} for Newspeak. See also \url{http://www.mirandabanda.org/cogblog/about-cog/}.} because of the way methods are looked up, and its syntax differs heavily from Smalltalk. For example, classes are not defined through \texttt{subclass:instanceVariableNames:} message sends, but through a new syntax which mimics a block structure known from Java or C++. The Hopscotch-based development environment~\cite{Bykov08hopscotch:towards} supports navigating nested classes and is optimized for Newspeak's way of defining classes, but looks completely different from the Squeak system browser. 

In contrast, \msname's class browser was designed to look similar to Squeak's system browser, so that it is easy to use for Squeak developers. We tried to limit the number of new concepts, such that code written in \msname should look familiar to Smalltalk developers. For example, there is no new syntax or user interface element for defining classes and instance variables. Instead, nested classes are defined through class generator methods which appear as and are in fact Smalltalk methods. We think that \msname exhibits a good balance between modularity and usability. After all, the goal of this project was not to implement a new modular programming language but to bring modularity concepts to Squeak.


%comparison with Newspeak: many ideas taken from it, but too complex
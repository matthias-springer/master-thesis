%
% abstract.tex
% Abstract
%

% BAMA-O (2009) §24.6 
%  (6) Die Abschlussarbeit ist eine für die Masterprüfung eigens
%  angefertigte Arbeit in deutscher Sprache. Mit Zustimmung der/des
%  Betreuerin/Betreuers kann die Arbeit auch in englischer Sprache abgefasst
%  werden. Erklären beide Gutachter/innen ihr Einverständnis, kann der
%  Prüfungsausschuss auch eine Anfertigung der Arbeit in einer anderen Sprache
%  zulassen. Ist die Arbeit in einer Fremdsprache verfasst, muss sie als Anhang
%  eine kurze Zusammenfassung in deutscher Sprache enthalten.
% BAMA-O (2013) § 30.12
%  (12) Die Masterarbeit ist eine Arbeit in deutscher Sprache, sofern die
%  fachspezifische Ordnung keine andere Sprache bestimmt. Mit Zustimmung der
%  Betreuerin bzw. des Betreuers kann die Arbeit auch in englischer Sprache
%  abgefasst werden. Erklären beide Prüfer ihr Einverständnis, kann der
%  Prüfungsausschuss auch eine Anfertigung der Arbeit in einer anderen Sprache
%  zulassen. Ist die Arbeit nicht in deutscher Sprache verfasst, muss sie als
%  Anhang eine kurze Zusammenfassung in deutscher Sprache enthalten.


\begin{abstract}
We present the concept, the implementation, and an evaluation of \msname, a module system for and written in Squeak/Smalltalk. \msname is inspired by Newspeak and based on class nesting: classes can be members of other classes, similarly to instance variables. 

Top-level classes (modules) are globals and nested classes can be accessed using message sends to the corresponding enclosing class. Class nesting effectively establishes a global and hierarchical namespace, and allows for modular decomposition, resulting in better understandability, if applied properly. 

Nested classes can be parameterized, allowing for external configuration of classes, a form of dependency injection. Furthermore, parameterized classes go hand in hand with mixin modularity. Mixins are a form of inter-class code reuse and based on single inheritance. Traits, another form of inter-class code reuse with explicit conflict resolution can be implemented on top of mixins.

We show how \msname can be used to solve the problem of duplicate classes in different modules, to provide a versioning and dependency management mechanism, and to improve understandability through hierarchical decomposition.
\end{abstract}

\begin{zusammenfassung}
  Die Zusammenfassung auf deutsch.
\end{zusammenfassung}

%%% Local Variables:
%%% mode: latex
%%% End:
